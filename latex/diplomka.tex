%!TEX TS-program = xelatex
%!TEX encoding = UTF-8 Unicode

\documentclass[12pt]{report}

\usepackage{tocloft}

%\setlength\cftparskip{-2pt}
%\setlength\cftbeforechapskip{0pt}

\usepackage{xecyr}
%\usepackage[utf8x]{inputenc}
\usepackage[russian]{babel}

%\usepackage[utf8x]{inputenc}
%\usepackage[english,russian]{babel}

\usepackage[top=2.5cm, bottom=2.5cm, left=3cm, right=2.5cm]{geometry}                % See geometry.pdf to learn the layout options. There are lots.
\geometry{a4paper}                   % ... or a4paper or a5paper or ... 
%\geometry{landscape}                % Activate for for rotated page geometry
%\usepackage[parfill]{parskip}    % Activate to begin paragraphs with an empty line rather than an indent
\usepackage{graphicx}
%\usepackage{amssymb}

\usepackage{fancyhdr}
\pagestyle{fancy}
\lhead{}
\rhead{\footnotesize{Система слежения за лицом человека}}
\cfoot{}
\rfoot{\thepage}

\pagenumbering{arabic}

\usepackage{setspace}
\onehalfspacing

\usepackage{fontspec,xltxtra,xunicode}
\defaultfontfeatures{Mapping=tex-text}
\setromanfont[Mapping=tex-text]{Times New Roman}
\setsansfont[Scale=MatchLowercase,Mapping=tex-text]{Gill Sans}
\setmonofont[Scale=MatchLowercase]{Andale Mono}

% Will Robertson's fontspec.sty can be used to simplify font choices.
% To experiment, open /Applications/Font Book to examine the fonts provided on Mac OS X,
% and change "Hoefler Text" to any of these choices.


\setlength{\parindent}{0cm}
\setlength{\parskip}{12pt plus 0.5ex minus 0.2ex}
\renewcommand{\headrulewidth}{0pt}

%\title{Искусственный интеллект в играх на примере игры "Войны планет"}
%\author{Александр Мочёнов}
%\date{26 Октября 2010 г.}                                           
% Activate to display a given date or no date

\usepackage{titlesec}
\titleformat{\chapter}[hang]{\bf\normalsize\uppercase}{\thechapter}{2pc}{}
\titleformat{\section}[hang]{\bf\normalsize}{\thesection}{2pc}{}	
\titleformat{\subsection}[hang]{\bf\normalsize}{\thesubsection}{2pc}{}
\titlespacing{\chapter}{0pt}{*4}{*1}

\usepackage{hyperref} 
%\usepackage{graphicx}
\usepackage{pstricks}
%\DeclareGraphicsExtensions{.png}

%\newcommand{\executeiffilenewer}[3]{%
% \ifnum\pdfstrcmp{\pdffilemoddate{#1}}%
% {\pdffilemoddate{#2}} >0 {\immediate\write18{#3}}\fi%
%}
%
%\newcommand{\includesvg}[1]{%
%\executeiffilenewer{#1.svg}{#1.pdf}%
%{inkscape -z -D --file=#1.svg --export-pdf=#1.pdf --export-latex}%
%\input{#1.tex}%
%}
\usepackage{subfig}

\usepackage{url}

\usepackage{natbib}
\bibpunct{(}{)}{;}{a}{,}{,}

\newenvironment{myItemize}{
	\begin{itemize}
  		\setlength{\itemsep}{1pt}
  		\setlength{\parskip}{0pt}
  		\setlength{\parsep}{0pt}
}{\end{itemize}}

\newenvironment{myEnumerate}{
	\begin{enumerate}
  		\setlength{\itemsep}{1pt}
  		\setlength{\parskip}{0pt}
  		\setlength{\parsep}{0pt}
}{\end{enumerate}}

\usepackage{totcount}
\regtotcounter{figure}
\regtotcounter{table}
\regtotcounter{chapter}

\usepackage[acronym,toc]{glossaries}
\makeglossaries
%\input{glossary}


\renewcommand{\cfttoctitlefont}{\normalsize\textbf}
\setlength{\cftaftertoctitleskip}{18pt}


\usepackage{listings}


\begin{document}
%\renewcommand{\chaptername}{} 
%\renewcommand{\thechapter}{} 
%\renewcommand{\thesection}{\arabic{section}} 

\begin{titlepage}
  \begin{center}
	\uppercase{Высшая школа майнор}\\*
	Институт инфотехнологии\\*
	Веб программирование\\[8cm]
	Александр Мочёнов\\*
	IT-3-Q-V-Tal\\[0.5cm]
	\large
	\textbf{Система слежения за лицом человека (на основе нейронной сети? (и облостей с кожным покровом?))}\\[1cm]
	\normalsize
	Дипломная работа\\[2cm]
	\begin{flushright}
		Руководитель:Jelena Faronova, MSc\\[7cm]
	\end{flushright}
	Таллинн 2010
  \end{center}
\end{titlepage}

\tableofcontents{\thispagestyle{fancyplain}}

\chapter*{Резюме}
\addcontentsline{toc}{chapter}{Резюме}
\thispagestyle{fancy}


%\printglossary[title=Терминалогия и переводы,style=fancy,toctitle=Терминалогия и переводы]

\chapter*{Введение}
\addcontentsline{toc}{chapter}{Введение}
\thispagestyle{fancy}

Цель, постановка Задач, гипотеза.

\chapter{Введение в предметную область}
\thispagestyle{fancy}

Теоретическая часть.

Про Computer Vision в целом.

Где и зачем применяется face detection?

* Обзор методов и решений. (способы face detection’a) (для каждого подпункт?) (С ROI, с цветом одежды, с отделением фона, с выделением движ. объектов)

* целые системы подобно реализуемой (голова робота и всё такое)

* про real-time

\chapter{Предлагаемый метод решения}
\thispagestyle{fancy}

Краткое описание всей системы. Диаграмма.

Модульность. Какие-то части могут быть реализованы по разному - но сами модули такие как тут.

используемые програмные библиотеки: OpenCV, PyBrain, ...

\section{Модуль нахождения лица}

\subsection{Нормализация контраста и баланс белого}
(предобработка)

Описание алгоритма.

Много примеров, гистограмм, псевдокод. (без особых результирующих картинок)

\subsection{Поиск зон с цветом кожного прокрова}
Общая информация о проблеме.

- зачем применяют 

	--про поиск картинок для взрослых
	
	--для face detetection
\subsubsection{Возможные пути решения}
Уже существующие различные методы нахождения цвета кожного покрова. Описание тут.
\subsubsection{Проблема выбора цветового пространства}
Про цветовые пространства. Про информативность каждого из них.
\subsubsection{Метод статического диапозона}
В работе реализуется он. Почему? (просто, быстро, достаточно эффективный)
Описание метода, псевдокод?

\subsection{Выделение и объединение областей с цветом кожного покрова}
Общии слова переходного характера. 
\subsubsection{Выделение найденных областей}
Описание алгоритма (сжатие, расширение) -  избавление от шума, более адекватные замкнутые области.
\subsubsection{Кластеризация}
Обоснование необходимости. Зачем объеденять.	

Почему обычный k-mean не подходит? примеры.

Описание метода кластеризации через минимальное оставное дерево. Что такое оставное дерево?

Примеры мест (ситуаций), где это необходимо. Где лицо состоит из нескольких небольших участков и полезно объеденять.

Псевдокод, диаграммы процесса.
\subsection{Фильтрация по пропорциям и заполненности}
Описание возможного постпроцессинга для отфильтровывания неподходящих участков.
\subsection{Классификация}
Описание проблеммы классификации в целом.
Опять о том какие методы бывают. О том, что сейчас применяют чаще.
\subsubsection{Выбор метода ИНС для классификации}
Почему выбрал ann? (real-time, простота понимания и использования)

Как это делают другие?
\subsubsection{Описание сети}
Несколько сетей для разных представлений. B/w, Edges

О проблемах недофитинга и overfit'инга.

Описание структуры ИНС. Почему именно такая.

bias'ы, преждевременная остановка, ...

Код с сосзданием сети.
\subsubsection{Обучение сети}
Первый этап.

Проблема и важность выбора примеров для обучения.

Применяемые базы лиц, усреднённые лица, возможная рамочка

Проблема выбора "не лиц".

Описание подготовки выборок для тренировки и тестирования.

Код PyBrain по тренеровки сети.
\subsubsection{Применение сети}
Сохранение и загрузка обученной сети.

Понятие порога.

sliding window алгоритм. диаграмы, код.

Кластеризация всех найденных лиц в группы, что бы отсечь случайные Flase positives. Overlap'ы и всё такое.
\section{Выбор цели для слежения}
найти наибольшее лицо

искать ближайшее к тому, за которым уже следим
\section{Механическая часть}
Работа с сервоприводами
\subsection{описание установки для демонстрации}
arduino,
сервоприводы,
камеры
\subsection{Подсчёт вектора движения}
\subsection{Arduino}
коммуникация c PC

листинги кода, диаграммы (этого нет =/ )





\chapter{Результаты работы (Испытания?)}
\section{Автоконтраст и баланс белого}
Когда работает? А когда не очень? примеры и того и того.

Возможные пути решения.
\section{Поиск зон с кожным покровом}
Когда работает? А когда не очень? примеры и того и того.

Пути решение. (Другой метод, выбор более узкой области диапазонов. - пример возможного приложения для сбора образцов)

О том что хорошо, что больше false negative, чем false positive
\section{Объединение областей}
Примеры хорошой и плохой работы.

Как можно улучшить. (выбор другого алгоритма выбора цвета кожи, подгонка параметров кластеризации)
\section{(Результаты) работа с ИНС}
\subsection{Различные представления}
Почему представление с пограничными областями не работает. Усреднённые морды где видно проблему. Как-то улучшить алгоритм выявления пограничных областей? Какие-то другие представления?
\subsection{(Результаты) обучения и тестирования}
Цифры, проценты результативности на тестовых данных. False positive, False negative. Примеры неузнанных лиц, примеры узнанных не лиц. Усреднённые нелица, усреднённые ненайденные лица.
Как можно улучшить?

- Правильная структура сети, выделяющая характеристики (features)

- Более тщательный подбор примеров (глаза на одном месте, одна ореинтация)

- Разные классы для разных поз (направление взгляда - прямо, вверх, вниз, вправо, влево)

- Икрементальный процесс обучения (где все falses из тестового набора добавляются обратно в набор обучения)
\subsection{Выбор лица и arduino}
Так и не успел закончить эту часть. Что писать в результатах пока не знаю.
\section{Испытание всей системы}
результаты испытаний.

небольшие ошибка на всех уровнях в итоге дают неудовлетворительный результат. улучшать необходимо каждый из элементов в отдельности.

- о проблеме 2-7 процентах на 97 000 примерах с одного кадра. Много false positive. Сложно настроить правильные порог.

- о проблеме со скорость. Решение - оптимизация и использование psyco.

\chapter*{Заключение и выводы}
\addcontentsline{toc}{chapter}{Заключение и выводы}
\thispagestyle{fancy}


\appendix
\chapter{Приложение. Отчёт по курсовой практике}
%\pagebreak
%-

%\clearpage
%\pagebreak
%-

%\clearpage
%\pagebreak
%-

%\clearpage
%\pagebreak
%-
%\clearpage


%\clearpage

\addcontentsline{toc}{chapter}{Литература}

\bibliographystyle{plainnat}
\bibliography{biblio}

\end{document}  